\documentclass[12pt, a4paper]{article}
\usepackage[UTF8]{ctex}
\usepackage{graphicx}
\usepackage{geometry}
\usepackage{titlesec}
\usepackage{enumitem}
\usepackage{hyperref}
\usepackage{xcolor} 


\geometry{a4paper, left=2.5cm, right=2.5cm, top=2.5cm, bottom=2.5cm}
\setlength{\parskip}{0.5em}
\renewcommand{\baselinestretch}{1.2}

\titleformat{\section}{\Large\bfseries}{\thesection}{1em}{}
\titleformat{\subsection}{\large\bfseries}{\thesubsection}{1em}{}
\titleformat{\subsubsection}{\bfseries}{\thesubsubsection}{1em}{}

\title{深圳大学 Python程序设计实验报告}
\author{学生姓名:刘辰 \\ 学号:2024401003 \\ 指导老师:樊超}
\date{提交日期:2025年12月10日(二次提交)}

\begin{document}

\maketitle

\section*{一、引言与技术全景理解}

\subsection*{1. 引言}
\begin{itemize}[leftmargin=*]
    \item \textbf{项目概要} \\
    游戏采用四层架构,分为输入/事件层、控制层、模型层和视图/UI层,确保了各部分的良好分离与协作。此外,游戏还实现了完整的用户认证系统、状态管理、碰撞检测及优化的核心游戏循环。
\end{itemize}

\subsection*{2. 技术全景理解}
\textbf{核心架构设计(见图1)} \\
项目采用了四层架构(输入/事件层 → 控制层 → 模型层 → 视图/UI层),其中控制层与视图、模型双向交互,避免了传统MVC架构的过度解耦,适合游戏开发的需求。配置参数、持久化数据和资源文件则被分别存储在 \texttt{settings.py}、\texttt{users.json} 和游戏资源文件中。


\textbf{状态管理的设计(见图2)} \\
使用栈式状态机来管理游戏状态,实现了流畅的状态切换,例如:菜单、游戏进行中、暂停和游戏结束等状态。此设计能够清晰地定义每个状态的转换规则,避免状态混乱。



\section*{二、关键技术分析与实现}

\subsection*{1. 用户认证系统}
\begin{itemize}[leftmargin=*]
    \item 游戏实现了一个轻量级的用户认证系统,包括注册、登录验证、密码加密(使用盐值哈希)和分数持久化功能。用户数据通过 \texttt{auth.py} 处理,并存储在 \texttt{users.json} 中。此外,还支持排行榜功能,能够通过 \texttt{get\_leaderboard()} 函数展示 Top-N 玩家成绩。
\end{itemize}

\subsection*{2. 核心游戏循环(见图3)}
游戏主循环采用了高效的设计,包括非阻塞的输入处理、状态判断与路由、渲染管道和性能控制。每个循环步骤保证了游戏的流畅运行,并且使用 \texttt{clock.tick(FPS)} 来确保稳定的帧率。

\subsection*{3. 碰撞检测系统}
\begin{itemize}[leftmargin=*]
    \item 碰撞检测系统使用了多级碰撞检测策略,包括子弹与外星人、飞船的碰撞检测。采用了 \texttt{pygame.sprite.groupcollide()} 进行批量检测,从而优化了性能,并且减少了无效的检测过程。
\end{itemize}


\begin{figure}[h]
  \centering
  \begin{minipage}[b]{0.24\textwidth}
    \centering
    \href{run:./images/architecture_full.png}{
      \includegraphics[width=\linewidth]{image1.png}
    }
    \caption{四层架构示意图\\\scriptsize\textcolor{blue}{点击图片查看大图}}
    \label{fig:img1}
  \end{minipage}
  \hfill
  \begin{minipage}[b]{0.24\textwidth}
    \centering
    \href{run:./images/statemachine_full.png}{
      \includegraphics[width=\linewidth]{image2.png}
    }
    \caption{状态机示意图\\\scriptsize\textcolor{blue}{点击图片查看大图}}
    \label{fig:img2}
  \end{minipage}
  \hfill
  \begin{minipage}[b]{0.24\textwidth}
    \centering
    \href{run:./images/gameloop_full.png}{
      \includegraphics[width=\linewidth]{image3.png}
    }
    \caption{游戏主循环\\\scriptsize\textcolor{blue}{点击图片查看大图}}
    \label{fig:img3}
  \end{minipage}
  \hfill
  \begin{minipage}[b]{0.24\textwidth}
    \centering
    \href{run:./images/gameplay_full.png}{
      \includegraphics[width=\linewidth]{game_screenshot.png}
    }
    \caption{游戏实物截图\\\scriptsize\textcolor{blue}{点击图片查看大图}}
    \label{fig:img4}
  \end{minipage}
\end{figure}




\section*{三、面向对象设计与优化策略}

\subsection*{1. 实体管理系统}
\begin{itemize}[leftmargin=*]
    \item 游戏中的实体管理采用面向对象设计,Alien、Boss、Bullet、Ship、PowerUp 等类通过继承自 \texttt{pygame.sprite.Sprite} 类来简化管理。除了继承体系外,还有专门的类如 Ship(玩家飞船)和 Fleet(外星人舰队)来负责不同实体的独立管理。
\end{itemize}

\subsection*{2. 性能优化}
通过批量渲染、对象池模式和事件驱动更新,显著提升了游戏的性能。子弹对象的复用、资源懒加载(按需加载音效与图片)以及减少每帧不必要的计算,都极大提高了游戏的流畅度。

\subsection*{3. 扩展性与维护性}
游戏架构具备良好的扩展性与维护性,支持新增游戏模式、敌人类型以及道具系统的扩展。此外,联机功能也可以通过扩展 \texttt{auth.py} 模块来支持,方便未来的功能拓展。

\subsection*{创新扩展功能}
\begin{itemize}[leftmargin=*]
    \item \textcolor{red}{多用户注册/登录 + 成绩记录系统}
    \item \textcolor{red}{完整 UI 组件库(按钮、输入框、菜单界面)}
    \item \textcolor{red}{集中式状态机框架,管理开始/暂停/战斗/结束等状态}
    \item \textcolor{red}{Boss 动画渲染 + 多阶段攻击模式}
    \item \textcolor{red}{随机道具掉落、效果判定与 Buff 实现}
    \item \textcolor{red}{游戏逻辑优化:帧缓存、事件循环处理、碰撞高效检测}
\end{itemize}

\section*{附录:项目源码链接}
\begin{itemize}[leftmargin=*]
    \item 项目源码:\url{https://github.com/santa3-1-1/Alien_invasion}
\end{itemize}

\section*{总结}
本项目展示了一个精心设计的游戏架构,不仅确保了游戏的流畅运行,还兼顾了代码的清晰与扩展性。四层架构的设计使得游戏的每个功能模块相对独立,便于开发与维护。通过状态机和碰撞检测的精妙设计,游戏的逻辑和交互变得更加清晰和高效。此外,性能优化的多方面考量也让游戏体验得到了提升。总体来看,这是一个非常有深度和可扩展性的游戏项目,适合用于学习与实践游戏开发中的核心概念。

\end{document}
